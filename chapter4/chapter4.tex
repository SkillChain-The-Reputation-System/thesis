\chapter{Hệ thống SkillChain}

\textbf{SkillChain} là một hệ thống ứng dụng mô hình uy tín phi tập trung dựa trên công nghệ chuỗi khối. Hệ thống cung cấp một môi trường làm việc xoay quanh các \textbf{thử thách}, cho phép người dùng tham gia để rèn luyện kỹ năng, trau dồi chuyên môn, củng cố kiến thức và đồng thời xây dựng hồ sơ uy tín một cách minh bạch, công khai và bất biến.
Bên cạnh đó, hệ thống thiết lập một cơ chế đánh giá công bằng, khách quan và được dẫn dắt bởi cộng đồng chuyên gia, nhằm đảm bảo rằng năng lực của người dùng được \textbf{ghi nhận đúng thực chất}, qua đó thúc đẩy động lực phát triển kỹ năng một cách bền vững và liên tục.

Ngoài việc hỗ trợ người học thể hiện năng lực thông qua hành động cụ thể, SkillChain còn cung cấp \textbf{công cụ xác thực đáng tin cậy} cho các nhà tuyển dụng trong việc đánh giá trình độ ứng viên --
vượt ra khỏi giới hạn của văn bằng, chứng chỉ hay khai báo chủ quan trong các mô hình tuyển dụng truyền thống.

\section{Yêu cầu chức năng chính}

\subsection{Chức năng của người dùng}

\textbf{Người dùng} là những cá nhân có nhu cầu xây dựng uy tín, chứng minh kỹ năng, trau dồi kiến thức và tìm kiếm cơ hội việc làm. 
Một người dùng phổ thông có thể thực hiện các chức năng sau:

\begin{itemize}
  \item \textbf{Truy cập và quản lý thông tin cá nhân, hồ sơ uy tín}: Người dùng có thể truy cập vào trang thông tin cá nhân, tại đây họ được phép cập nhật các thông tin liên quan đến hồ sơ người dùng.
        Đồng thời, người dùng có thể xem chi tiết hồ sơ uy tín của bản thân, bao gồm điểm uy tín toàn cục và điểm uy tín chuyên môn theo từng lĩnh vực, tương ứng với các loại thử thách đã được liệt kê tại mục 3.1.
  \item \textbf{Xem danh sách thử thách đang hiện hành trên hệ thống}: Người dùng có thể xem danh sách các thử thách đang có trên hệ thống, bao gồm cả thông tin chi tiết về thử thách và người đóng góp tương ứng.
  \item \textbf{Tham gia thực hiện thử thách}: Sau khi lựa chọn một thử thách phù hợp, người dùng sẽ phải trả một khoản token cho người đóng góp để được tham gia (khoản token sẽ được tính toán bởi hệ thống). Hệ thống sẽ cung cấp một không gian làm việc riêng nhằm giải quyết yêu cầu của thử thách và cho phép tùy chỉnh giải pháp.
        Tuy nhiên, người đóng góp sẽ không thể tham gia thực hiện chính thử thách do mình đóng góp.
  \item \textbf{Nộp và gửi giải pháp}: Sau khi hoàn thành, người dùng có thể nộp và gửi giải pháp lên hệ thống để được đánh giá. Lúc này, giải pháp sẽ chuyển sang trạng thái \textit{chờ đánh giá} và không thể chỉnh sửa thêm.
        Bản thân người dùng không thể tham gia đánh giá giải pháp của chính mình.
  \item \textbf{Xem danh sách thử thách đã và đang thực hiện}: Người dùng có thể xem danh sách các thử thách mà mình đã hoặc đang tham gia cùng với các giải pháp tương ứng.
        Đối với những thử thách đã được đánh giá, hệ thống sẽ hiển thị danh sách người đánh giá và điểm số chi tiết của từng người.
  \item \textbf{Xem các bài tuyển dụng đang hiện hành}: Người dùng có thể xem danh sách các bài đăng tuyển dụng đang được đăng tải bởi các nhà tuyển dụng trên hệ thống, bao gồm đầy đủ thông tin chi tiết của từng bài đăng.
  \item \textbf{Tham gia ứng tuyển và phỏng vấn}: Người dùng có thể nộp đơn ứng tuyển cho các vị trí tuyển dụng nếu đáp ứng yêu cầu đề ra trong bài đăng. 
  Sau khi ứng tuyển, người dùng sẽ chờ phản hồi từ phía nhà tuyển dụng. Trong trường hợp được chọn vào danh sách rút gọn, người dùng có thể tham gia phỏng vấn trực tuyến khi được nhà tuyển dụng lên lịch.
  \item \textbf{Xem danh sách các bài tuyển dụng đã và đang ứng tuyển}: Người dùng có thể theo dõi các bài tuyển dụng mà mình đã hoặc đang ứng tuyển, kèm theo trạng thái hiện tại của từng hồ sơ ứng tuyển (ví dụ: đang chờ phản hồi, đã được mời phỏng vấn, bị từ chối, v.v.).
\end{itemize}

Hệ thống còn mở rộng thêm các chức năng chuyên biệt theo vai trò cụ thể mà một người dùng phổ thông có thể sở hữu.

\subsubsection{Người đóng góp}

\textbf{Người đóng góp} là người dùng có vai trò tạo và thiết kế thử thách, đóng góp chúng vào ngân hàng thử thách chung của hệ thống. Một người đóng góp có thể thực hiện các chức năng sau:

\begin{itemize}
  \item \textbf{Tạo thử thách mới ở trạng thái nháp}: Mỗi thử thách bao gồm các thông tin bắt buộc như tiêu đề, loại thử thách, mô tả chi tiết, và mức token treo thưởng.
        Khi tạo thử thách, hệ thống cung cấp một không gian làm việc cho người đóng góp, cho phép chỉnh sửa nội dung trong giai đoạn nháp.
  \item \textbf{Gửi thử thách để kiểm duyệt}: Sau khi hoàn thiện nội dung, người đóng góp có thể gửi thử thách lên hệ thống để kiểm duyệt. Từ thời điểm này, thử thách chuyển sang trạng thái \textit{chờ kiểm duyệt} và không còn được chỉnh sửa.
        Người đóng góp không thể tham gia vào quá trình kiểm duyệt thử thách của chính mình.
  \item \textbf{Xem danh sách thử thách đã tạo}: Người đóng góp có thể xem danh sách các thử thách mà mình đã đóng góp. Đối với các thử thách đã được kiểm duyệt, có thể xem thông tin người kiểm duyệt, điểm đánh giá và mức thưởng token của từng người.
        Nếu thử thách đã được chấp nhận, người đóng góp cũng có thể xem danh sách người tham gia cùng với phần lợi nhuận thu được từ thử thách.
\end{itemize}

\subsubsection{Người kiểm duyệt}

\textbf{Người kiểm duyệt} là người dùng có vai trò đánh giá và phê duyệt chất lượng các thử thách được gửi từ phía người đóng góp. Một người kiểm duyệt có thể thực hiện các chức năng sau:

\begin{itemize}
  \item \textbf{Xem danh sách thử thách đang chờ kiểm duyệt}: Người kiểm duyệt có thể xem danh sách các thử thách đang ở trạng thái chờ phê duyệt, bao gồm thông tin chi tiết về thử thách và người đóng góp tương ứng.
  \item \textbf{Tham gia kiểm duyệt thử thách}: Sau khi lựa chọn một thử thách phù hợp, người kiểm duyệt có thể đăng ký tham gia vào quá trình đánh giá. Hệ thống sẽ cung cấp một không gian làm việc riêng để thực hiện quá trình kiểm duyệt.
  \item \textbf{Thực hiện kiểm duyệt thử thách}: Trong không gian làm việc, người kiểm duyệt có thể đánh giá nội dung thử thách và chấm điểm theo thang chuẩn của hệ thống. Sau khi gửi đánh giá, kết quả sẽ được ghi nhận vĩnh viễn và không thể chỉnh sửa.
  \item \textbf{Xem danh sách thử thách đã và đang kiểm duyệt}: Người kiểm duyệt có thể theo dõi danh sách các thử thách mà mình đã hoặc đang tham gia đánh giá.
        Đối với các thử thách đã hoàn tất kiểm duyệt, hệ thống sẽ hiển thị thông tin về các kiểm duyệt viên khác, điểm số đánh giá cũng như phần thưởng token được phân bổ cho từng người.
\end{itemize}

\subsubsection{Người đánh giá}

\textbf{Người đánh giá} là những người dùng có vai trò đánh giá và chấm điểm các giải pháp được gửi từ người dùng nói chung. Một người đánh giá có thể thực hiện các chức năng sau:

\begin{itemize}
  \item \textbf{Xem danh sách giải pháp đang chờ đánh giá}: Người đánh giá có thể xem danh sách các giải pháp đang ở trạng thái \textit{chờ đánh giá}. Tại thời điểm này, họ chỉ có thể xem nội dung thử thách và thông tin người gửi, nhưng chưa thể truy cập nội dung giải pháp.
  \item \textbf{Tham gia đánh giá giải pháp}: Sau khi lựa chọn một giải pháp phù hợp, người đánh giá có thể đăng ký tham gia vào quá trình đánh giá.
        Hệ thống sẽ cung cấp một không gian làm việc riêng để tiến hành việc đánh giá. Tuy nhiên, người đóng góp của thử thách không thể tham gia đánh giá giải pháp.
  \item \textbf{Thực hiện đánh giá giải pháp}: Trong không gian làm việc, người đánh giá sẽ được quyền xem nội dung giải pháp và tiến hành chấm điểm. Sau khi gửi đánh giá, kết quả sẽ được ghi nhận vĩnh viễn và không thể chỉnh sửa.
  \item \textbf{Xem danh sách giải pháp đã và đang đánh giá}: Người đánh giá có thể theo dõi danh sách các giải pháp mà mình đã hoặc đang tham gia đánh giá.
        Đối với các thử thách đã hoàn tất quá trình đánh giá, hệ thống sẽ hiển thị thông tin về các người đánh giá khác và điểm số tương ứng.
\end{itemize}

\subsection{Chức năng của nhà tuyển dụng}

\textbf{Nhà tuyển dụng} là chủ thể có nhu cầu tìm kiếm ứng viên, xác thực kỹ năng và đánh giá năng lực thông qua mô hình uy tín phi tập trung của hệ thống SkillChain. 
Một nhà tuyển dụng có thể thực hiện các chức năng sau:

\begin{itemize}
  \item \textbf{Tạo bài tuyển dụng ở trạng thái nháp}: nhà tuyển dụng có thể khởi tạo một bài đăng tuyển dụng với các thông tin bắt buộc như: tiêu đề, mô tả công việc, yêu cầu công việc, khoảng thời gian làm việc, mức lương, thời điểm kết thúc ứng tuyển, và ngưỡng yêu cầu về uy tín chuyên môn. 
  Ngoài ra, có thể bổ sung các thông tin tùy chọn như địa điểm làm việc hoặc yêu cầu về uy tín toàn cục. Bài đăng có thể được chỉnh sửa ở bất kỳ trạng thái nào.
  \item \textbf{Đăng bài tuyển dụng}: Sau khi hoàn thiện nội dung, nhà tuyển dụng có thể đăng bài lên hệ thống, qua đó cho phép người dùng đủ điều kiện ứng tuyển.
  \item \textbf{Xem thông tin ứng tuyển}: Đối với mỗi bài tuyển dụng, nhà tuyển dụng có thể xem danh sách các ứng viên đã ứng tuyển, bao gồm thông tin cá nhân và các chỉ số uy tín phù hợp với yêu cầu đề ra. 
  Sau khi đánh giá, nhà tuyển dụng có thể thay đổi trạng thái ứng tuyển của từng ứng viên (ví dụ: từ ``đang chờ'' sang ``đã đánh giá'').
  \item \textbf{Lập lịch phỏng vấn}: Đối với các ứng viên nằm trong danh sách rút gọn, nhà tuyển dụng có thể tạo lịch hẹn phỏng vấn dưới hình thức buổi họp trực tuyến để đánh giá sâu hơn trước khi đưa ra quyết định cuối cùng.
\end{itemize}

\subsection{Chức năng của hệ thống}

Bên cạnh các chức năng được phân tách theo vai trò người dùng, hệ thống SkillChain cũng đảm nhận một tập hợp các chức năng nền tảng, đóng vai trò duy trì toàn bộ quá trình vận hành, lưu trữ, đánh giá và phân phối thông tin trong nền tảng. Cụ thể, hệ thống chịu trách nhiệm thực hiện các tác vụ sau:

\begin{itemize}
  \item \textbf{Lưu trữ và cập nhật dữ liệu}: Hệ thống có nhiệm vụ lưu trữ các dữ liệu do người dùng và nhà tuyển dụng tạo ra (bao gồm: thông tin cá nhân, thử thách, giải pháp, bài tuyển dụng, v.v.) trên cả nền tảng on-chain và off-chain. Bên cạnh đó, hệ thống cũng chịu trách nhiệm cập nhật các dữ liệu khi có thay đổi hợp lệ từ phía người dùng.
  \item \textbf{Truy xuất dữ liệu theo yêu cầu}: Hệ thống xử lý các yêu cầu truy vấn từ người dùng và nhà tuyển dụng, bao gồm việc lấy dữ liệu từ blockchain hoặc từ các nguồn lưu trữ off-chain, nhằm hiển thị thông tin một cách nhanh chóng và chính xác.
  \item \textbf{Quản lý phiên kiểm duyệt thử thách}: Hệ thống thu thập điểm kiểm duyệt từ tất cả các người kiểm duyệt tham gia một phiên kiểm duyệt, tính toán điểm chất lượng cuối cùng của thử thách, từ đó phân bổ phần thưởng token phù hợp và cập nhật điểm uy tín của từng người kiểm duyệt và người đóng góp.
  \item \textbf{Tính toán khoản token tham gia thử thách}: Hệ thống xác định mức phí token mà người tham gia cần chi trả để tham gia một thử thách, dựa trên thông tin cấu hình thử thách.
  \item \textbf{Quản lý phiên đánh giá giải pháp}: Hệ thống tổng hợp các điểm đánh giá từ các người đánh giá, tính toán điểm trung bình cuối cùng của giải pháp, và từ đó cập nhật điểm uy tín của cả người thực hiện giải pháp và người đánh giá.
  \item \textbf{Thống kê quá trình tuyển dụng}: Hệ thống tổng hợp dữ liệu từ các bài đăng, hồ sơ ứng tuyển, lịch phỏng vấn và kết quả tuyển dụng để cung cấp báo cáo thống kê cho nhà tuyển dụng, giúp đánh giá hiệu quả và chất lượng quy trình tuyển dụng của họ.
\end{itemize}

\section{Yêu cầu phi chức năng}

Bên cạnh các chức năng chính, hệ thống SkillChain cần đáp ứng một số yêu cầu phi chức năng để đảm bảo hiệu suất, khả năng sử dụng, bảo mật và độ tin cậy trong quá trình vận hành. Các yêu cầu này được liệt kê như sau:

\begin{itemize}
  \item \textbf{Tính minh bạch và bất biến}: Mọi hành vi quan trọng của người dùng như đóng góp thử thách, tham gia giải thử thách, đánh giá giải pháp hay kiểm duyệt chất lượng thử thách đều phải được ghi nhận on-chain nhằm đảm bảo tính minh bạch, công khai và không thể chỉnh sửa sau khi ghi nhận.
  \item \textbf{Tính bảo mật}: Thông tin tài khoản người dùng và dữ liệu nhạy cảm (ví dụ: dữ liệu giải pháp lưu trên nền tảng off-chain) cần được bảo vệ khỏi truy cập trái phép. Hệ thống phải xác thực danh tính người dùng thông qua ví điện tử và chỉ cấp quyền tương ứng nếu có yêu cầu truy cập.
  \item \textbf{Tính khả dụng}: Hệ thống phải đảm bảo hoạt động liên tục, sẵn sàng phục vụ người dùng 24/7. Trong trường hợp một số dịch vụ off-chain bị gián đoạn, hệ thống vẫn cần đảm bảo dữ liệu on-chain có thể được truy xuất độc lập.
  \item \textbf{Khả năng mở rộng}: Hệ thống cần thiết kế để có thể mở rộng số lượng và số loại thử thách, người dùng và bài tuyển dụng trong tương lai mà không ảnh hưởng đến hiệu năng hoặc trải nghiệm người dùng.
  \item \textbf{Khả năng tương thích}: Ứng dụng cần tương thích với các trình duyệt phổ biến (Chrome, Firefox, Safari), ví điện tử (MetaMask, CoinBase) và nhiều loại thiết bị điện tử (máy tính, máy tính bảng, điện thoại thông minh).
  \item \textbf{Khả năng sử dụng}: Giao diện người dùng cần trực quan, rõ ràng và dễ sử dụng đối với người dùng mới, đồng thời hỗ trợ đa vai trò với trải nghiệm nhất quán.
  \item \textbf{Hiệu suất hệ thống}: Thời gian phản hồi cho các thao tác phổ biến như truy xuất danh sách thử thách, nộp giải pháp, hay đánh giá thử thách không được vượt quá 3 giây trong điều kiện bình thường. Hệ thống cần tối ưu hóa việc đồng bộ dữ liệu giữa blockchain và phần giao diện.
  \item \textbf{Khả năng kiểm thử và bảo trì}: Hệ thống cần được xây dựng theo hướng mô-đun, dễ dàng kiểm thử và nâng cấp. Các thành phần frontend, backend, smart contract và dữ liệu lưu trữ cần được tách biệt rõ ràng.
\end{itemize}

\section{Kiến trúc tổng thể}

\section{Chi tiết công nghệ sử dụng}

\section{Thiết kế các thành phần chính}

\section{Triển khai và thử nghiệm}