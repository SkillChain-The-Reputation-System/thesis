\chapter{Kết luận}

\section{Mục tiêu của đề tài}

Mục tiêu của đề tài là thiết kế và xây dựng một hệ thống đánh giá kỹ năng phi tập trung có tên SkillChain, ứng dụng công nghệ blockchain để lưu trữ và xác thực các hoạt động học tập, đóng góp và đánh giá trong môi trường học tập cộng đồng.
Hệ thống cho phép người dùng tham gia giải các thử thách thực tế, nhận phản hồi từ các chuyên gia, đồng thời tích lũy chỉ số uy tín chuyên môn làm cơ sở cho việc phát triển năng lực cá nhân và tiếp cận các cơ hội nghề nghiệp.
Ngoài ra, hệ thống còn hỗ trợ nhà tuyển dụng khai thác dữ liệu uy tín và quá trình học tập của ứng viên để phục vụ cho hoạt động tuyển dụng minh bạch và hiệu quả.

\section{Phương pháp thực hiện}

Để đạt được mục tiêu đề tài, chúng tôi đã áp dụng phương pháp thiết kế và triển khai hệ thống theo hướng kết hợp các công nghệ phi tập trung, lấy blockchain làm nền tảng lưu trữ minh bạch và không thể thay đổi. Cụ thể:
\begin{itemize}
  \item \textbf{Xác định yêu cầu và thiết kế mô hình hệ thống}: Dựa trên vấn đề thực tế về việc xác thực kỹ năng và hồ sơ cá nhân trong lĩnh vực Công nghệ thông tin, nhóm đã xây dựng một mô hình hệ thống uy tín phi tập trung.
        Mô hình được phân tích và thiết kế theo hướng đa tác nhân, trong đó mỗi vai trò (người học, người đóng góp, người kiểm duyệt, người đánh giá và nhà tuyển dụng) đều có hành vi và đặc quyền riêng.
  \item \textbf{Ứng dụng công nghệ phù hợp}:
        \begin{itemize}
          \item \textbf{Blockchain và Smart Contract} được sử dụng để lưu trữ các thông tin quan trọng như hồ sơ người dùng, thử thách, giải pháp, kết quả đánh giá, và chỉ số uy tín.
          \item \textbf{Dịch vụ lưu trữ phi tập trung} được tích hợp để lưu nội dung dung lượng lớn như mô tả thử thách, giải pháp của người dùng, và thông tin hồ sơ mở rộng.
          \item \textbf{Frontend và Backend} được phát triển với giao diện thân thiện, hỗ trợ đầy đủ các tác vụ cho người dùng tương tác, đồng thời tích hợp với hợp đồng thông minh để đồng bộ dữ liệu và hành vi.
        \end{itemize}
  \item \textbf{Mô hình hóa uy tín}: Đề tài xây dựng một hệ thống tính điểm uy tín theo từng vai trò và lĩnh vực chuyên môn, áp dụng các công thức tính điểm thưởng/phạt dựa trên độ chính xác, mức độ đóng góp và sự đồng thuận của cộng đồng. Uy tín được phân thành uy tín chuyên môn và uy tín toàn cục, lưu trữ minh bạch on-chain, và được sử dụng làm căn cứ phân quyền hành vi.
  \item \textbf{Phát triển hệ thống minh họa}: Cuối cùng, chúng tôi đã hiện thực hóa mô hình dưới dạng một hệ thống hoạt động ở môi trường cục bộ, có đầy đủ các luồng đóng góp thử thách, kiểm duyệt, tham gia và đánh giá thử thách, cùng tính năng hỗ trợ tuyển dụng.
\end{itemize}

\section{Tóm tắt kết quả đạt được}

Trong khuôn khổ đề tài, nhóm đã đề xuất và hiện thực hóa thành công một hệ thống ứng dụng phi tập trung mang tên SkillChain, hướng đến việc xác thực và tích lũy uy tín chuyên môn cho người học thông qua cơ chế đóng góp -- đánh giá -- kiểm duyệt cộng đồng.
Các kết quả cụ thể đã đạt được gồm:
\begin{itemize}
  \item \textbf{Phát triển thành công hệ thống Smart Contract} gồm các hợp đồng quản lý thử thách, giải pháp, uy tín, chi phí và tuyển dụng, đảm bảo thực hiện đúng các quy trình logic phức tạp như đánh giá, kiểm duyệt và cập nhật uy tín người dùng.
  \item \textbf{Thiết kế mô hình dữ liệu và vai trò} phù hợp với các đối tượng trong hệ thống: người học, người đóng góp, người kiểm duyệt, người đánh giá và nhà tuyển dụng. Các vai trò được kiểm soát bằng chỉ số uy tín và được phân quyền rõ ràng.
  \item \textbf{Xây dựng giao diện người dùng} đầy đủ cho các tác vụ chính: tạo thử thách, kiểm duyệt thử thách, giải quyết thử thách, đánh giá giải pháp, đăng tuyển dụng và quản lý ứng viên.
  \item \textbf{Tích hợp lưu trữ phi tập trung} để bảo đảm tính công khai và bất biến cho dữ liệu có dung lượng lớn, đồng thời kết hợp với dữ liệu on-chain để duy trì khả năng xác minh và truy xuất minh bạch.
  \item \textbf{Xây dựng thành công một hệ thống mẫu} có khả năng hoạt động đầy đủ trong môi trường kiểm thử cục bộ, giúp kiểm chứng ý tưởng và cơ chế của hệ thống.
\end{itemize}
Các kết quả đạt được không chỉ chứng minh tính khả thi của mô hình đề xuất, mà còn mở ra tiềm năng áp dụng trong thực tiễn, nơi mà tính minh bạch, uy tín và khuyến khích đóng góp cộng đồng là yếu tố then chốt cho môi trường học tập và tuyển dụng phi tập trung.

\section{Những hạn chế và giới hạn của đề tài}

Mặc dù đề tài đã đạt được nhiều kết quả tích cực trong việc xây dựng một hệ thống nguyên mẫu hoạt động đầy đủ, vẫn còn tồn tại một số hạn chế và giới hạn như sau
\subsection{Đối với hệ thống}
\begin{itemize}
  \item \textbf{Thiếu giới hạn thời gian trong quá trình kiểm duyệt}: Hệ thống không quy định thời hạn nộp kết quả kiểm duyệt, dẫn đến nguy cơ trì hoãn hoặc bỏ dở quy trình.
  \item \textbf{Bộ câu hỏi kiểm duyệt còn quá chung}: Hiện tại, bộ câu hỏi kiểm duyệt chưa được thiết kế riêng cho từng lĩnh vực chuyên môn, dẫn đến việc đánh giá thiếu chính xác hoặc không phản ánh đúng yêu cầu đặc thù của từng lĩnh vực.
  \item \textbf{Không hỗ trợ phản hồi kiểm duyệt}: Người kiểm duyệt không thể để lại nhận xét hay góp ý cho người đóng góp, hạn chế tính tương tác hai chiều.
  \item \textbf{Tiêu chí chấm điểm giải pháp còn đơn giản}: Việc chấm điểm chỉ dựa trên điểm số tổng quát, chưa phản ánh đầy đủ các khía cạnh đánh giá chi tiết.
  \item \textbf{Thiếu giới hạn thời gian trong quá trình đánh giá}: Tương tự như kiểm duyệt, không có thời hạn đánh giá dẫn đến nguy cơ đình trệ toàn bộ quy trình.
  \item \textbf{Không hỗ trợ phản hồi từ người đánh giá}: Người học không nhận được nhận xét cụ thể về giải pháp đã nộp, ảnh hưởng đến khả năng cải thiện kỹ năng.
  \item \textbf{Thiếu tính năng bảo mật nâng cao}: Dữ liệu nhạy cảm như giải pháp, đánh giá và kiểm duyệt chưa được mã hóa hoặc kiểm soát truy cập đầy đủ, tiềm ẩn rủi ro an ninh thông tin.
  \item \textbf{Phạm vi lĩnh vực chuyên môn còn hạn chế}: Các lĩnh vực chuyên môn trong hệ thống hiện vẫn mang tính khái quát, chưa phân tách đầy đủ các chuyên ngành nhỏ trong lĩnh vực Công nghệ thông tin.
        Điều này có thể ảnh hưởng đến độ chính xác khi xác định uy tín chuyên môn và phân quyền vai trò tương ứng.
  \item \textbf{Thiếu cơ chế kiểm soát danh tính trong kiểm duyệt và đánh giá}: Hệ thống chưa đảm bảo rằng những người tham gia kiểm duyệt hoặc đánh giá là những cá nhân khác nhau, dẫn đến khả năng một người dùng sử dụng nhiều ví điện tử để tham gia nhiều vai trò, làm giảm tính khách quan và độ tin cậy của kết quả.
  \item \textbf{Chưa kiểm soát hành vi của người kiểm duyệt và người đánh giá}: Hệ thống chưa có cơ chế giám sát hoặc chế tài đối với các hành vi thiên vị, phá hoại hoặc không công bằng từ phía người kiểm duyệt và người đánh giá.
\end{itemize}

\subsection{Đối với giao diện người dùng}
\begin{itemize}
  \item \textbf{Chưa hỗ trợ tra cứu hồ sơ người dùng khác}: Hệ thống hiện chưa cho phép người dùng tra cứu hồ sơ uy tín hoặc thông tin cá nhân của người dùng khác (ngoại trừ nhà tuyển dụng), hạn chế khả năng tương tác và kết nối cộng đồng.
  \item \textbf{Chưa ràng buộc đăng ký thông tin cá nhân khi thực hiện hành động quan trọng}: Người dùng chưa hoàn tất đăng ký thông tin cá nhân vẫn có thể thực hiện một số tác vụ quan trọng nếu đạt yêu cầu về uy tín, ảnh hưởng đến tính minh bạch và kiểm soát.
  \item \textbf{Mô tả thử thách chưa tối ưu hóa}:
        \begin{itemize}
          \item Thiếu hỗ trợ biểu thức toán học và ký hiệu đặc biệt trong phần mô tả, gây bất tiện cho việc trình bày thuật toán.
          \item Thiếu hỗ trợ tải tập tin, buộc người dùng phải sử dụng liên kết ngoài, gây gián đoạn trải nghiệm.
        \end{itemize}
  \item \textbf{Không gian trình bày giải pháp còn hạn chế}: Hiện tại chỉ hỗ trợ văn bản thuần túy, thiếu khả năng chèn công thức hoặc tệp đính kèm.
  \item \textbf{Thiếu kiểm soát vai trò nhà tuyển dụng}: Bất kỳ người dùng nào có ví điện tử đều có thể truy cập không gian nhà tuyển dụng mà không có xác thực về uy tín.
\end{itemize}

\section{Hướng phát triển tiếp theo}

Dựa trên những kết quả đạt được và các hạn chế còn tồn tại, đề tài có thể được tiếp tục phát triển theo các hướng sau.

\subsection{Giới hạn lại chỉ số uy tín}

Hiện tại, chỉ số uy tín chuyên môn hoặc uy tín toàn cục của một cá nhân có thể tăng không giới hạn, dẫn đến khó khăn trong việc phân loại mức độ uy tín giữa các người dùng -- đặc biệt là đối với những người kiểm duyệt có chỉ số uy tín vượt quá 200.

Để khắc phục vấn đề này, chúng tôi đề xuất giới hạn chỉ số uy tín và điều chỉnh các ngưỡng phân quyền vai trò như sau:
\begin{itemize}
  \item Giới hạn chỉ số uy tín chuyên môn và toàn cục ở mức tối đa là 100 nhằm đảm bảo tính ổn định, dễ quản lý và dễ phân loại cấp bậc người dùng.
  \item Các ngưỡng điểm uy tín để đạt được từng vai trò được điều chỉnh lại như bảng~\ref{tab:suggested-role-reputation-threshold-adjustment}.
\end{itemize}

\begin{table}[H]
  \centering
  \begin{tabular}{|c|c|}
    \hline
    \textbf{Vai trò} & $R_{\text{threshold}}$ \\ \hline
    Người đóng góp   & 50                     \\ \hline
    Người đánh giá   & 70                     \\ \hline
    Người kiểm duyệt & 90                     \\ \hline
  \end{tabular}
  \caption{Bảng đề xuất điều chỉnh ngưỡng điểm uy tín theo vai trò}
  \label{tab:suggested-role-reputation-threshold-adjustment}
\end{table}

\subsection{Xây dựng bộ câu hỏi kiểm duyệt và đánh giá cho từng lĩnh vực chuyên môn}

Hiện tại, hệ thống chỉ áp dụng một bộ câu hỏi kiểm duyệt chung và thang điểm đánh giá là 100 cho tất cả các lĩnh vực chuyên môn. Tuy nhiên, mỗi lĩnh vực lại có những đặc thù riêng biệt về kiến thức, kỹ năng và tiêu chuẩn chất lượng.
Việc sử dụng chung một bộ tiêu chí đánh giá cho mọi loại thử thách là chưa phù hợp, có thể làm giảm độ chính xác và khách quan trong quá trình kiểm duyệt và đánh giá.

Do đó, chúng tôi đề xuất xây dựng các bộ câu hỏi kiểm duyệt và đánh giá chuyên biệt tương ứng với từng lĩnh vực chuyên môn. Mỗi bộ câu hỏi sẽ được thiết kế nhằm phản ánh đúng bản chất chuyên môn của lĩnh vực đó, đảm bảo tính toàn diện, công bằng và nhất quán trong quá trình đánh giá.
Các bộ câu hỏi này sẽ được xây dựng và hiệu chỉnh bởi các chuyên gia có chuyên môn sâu trong từng lĩnh vực, đồng thời sẽ được rà soát, cập nhật định kỳ để đảm bảo tính thời sự và phù hợp với xu hướng công nghệ hiện nay.

\subsection{Thiết lập hội đồng kiểm soát hành vi}

Hiện tại, người kiểm duyệt và người đánh giá có thể thực hiện các hành vi ảnh hưởng trực tiếp đến kết quả thử thách và giải pháp mà không chịu bất kỳ sự giám sát nào, dẫn đến nguy cơ xuất hiện các hành vi không công bằng hoặc thiên vị.

Để khắc phục vấn đề này, chúng tôi đề xuất thành lập các hội đồng kiểm soát hành vi riêng biệt cho từng lĩnh vực chuyên môn. Cụ thể, mỗi lĩnh vực sẽ có:
\begin{itemize}
  \item Một \textbf{hội đồng kiểm soát kiểm duyệt} để đánh giá hành vi của người kiểm duyệt.
  \item Một \textbf{hội đồng kiểm soát đánh giá} để giám sát hành vi của người đánh giá.
\end{itemize}

\subsubsection{Cơ chế kiểm soát hành vi}

Sau khi tất cả người kiểm duyệt hoặc người đánh giá nộp kết quả cho một thử thách hoặc giải pháp, hệ thống sẽ \textbf{ngẫu nhiên chọn một thành viên} từ hội đồng kiểm soát tương ứng để đánh giá hành vi trong phiên này.

Thành viên được chọn sẽ xem xét toàn bộ nội dung đánh giá và xác định từng bài đánh giá là \textbf{``hợp lý''} hay \textbf{``không hợp lý''}, mà không xét ai ``đúng hơn''.
Mục tiêu là phát hiện các đánh giá có dấu hiệu thiếu công bằng, sai lệch hoặc thiếu khách quan.
Những bài đánh giá bị đánh giá là ``không hợp lý'' sẽ:
\begin{itemize}
  \item Bị loại khỏi quá trình tổng hợp kết quả.
  \item Khiến người kiểm duyệt hoặc người đánh giá \textbf{giảm điểm uy tín}, riêng người kiểm duyệt cũng sẽ \textbf{không được thưởng}.
\end{itemize}

\subsubsection{Hình thức tuyển chọn thành viên hội đồng kiểm soát}

Để đảm bảo quy mô và năng lực đánh giá, các hội đồng kiểm soát cần được mở rộng tương ứng với số lượng thử thách và giải pháp phát sinh. Cơ chế tuyển chọn thành viên hội đồng như sau:

\begin{itemize}
  \item Người kiểm duyệt hoặc người đánh giá có uy tín chuyên môn đạt ngưỡng tối thiểu sẽ được phép đăng ký tham gia hội đồng kiểm soát tương ứng với lĩnh vực chuyên môn.
  \item Hội đồng hiện tại sẽ gửi bộ câu hỏi tuyển chọn chuyên biệt để người đăng ký trả lời.
  \item Nếu câu trả lời vượt ngưỡng yêu cầu nhất định (ví dụ, câu trả lời được hơn 90\% thành viên hội đồng hiện tại chấp thuận), cá nhân đó sẽ được kết nạp vào hội đồng.
\end{itemize}

\textbf{Giai đoạn khởi tạo ban đầu:} Hệ thống sẽ mời một chuyên gia đảm nhiệm vai trò \textit{quản lý chuyên môn} cho mỗi lĩnh vực. Người này có nhiệm vụ lựa chọn từ 3--5 chuyên gia đáng tin cậy để hình thành các hội đồng kiểm soát đầu tiên.
Quy trình này không dựa vào điểm uy tín trên hệ thống, mà dựa vào kinh nghiệm và đánh giá chuyên môn thực tế.

\subsubsection{Quản lý bộ câu hỏi kiểm duyệt, đánh giá và tuyển chọn}

Các hội đồng kiểm soát có nhiệm vụ xây dựng và duy trì:
\begin{itemize}
  \item Bộ câu hỏi dùng để kiểm duyệt thử thách và đánh giá giải pháp.
  \item Bộ câu hỏi dùng để tuyển chọn thành viên mới vào hội đồng kiểm soát.
\end{itemize}

Tại mỗi phiên họp định kỳ:
\begin{itemize}
  \item Mỗi thành viên có thể đề xuất chỉnh sửa, bổ sung hoặc loại bỏ câu hỏi.
  \item Mỗi đề xuất sẽ được hội đồng biểu quyết ngang phiếu.
  \item Nếu tỷ lệ đồng thuận vượt ngưỡng yêu cầu nhất định (ví dụ: đề xuất chỉnh sửa câu hỏi K được 80\% thành viên hội đồng đồng ý), đề xuất sẽ được thông qua.
\end{itemize}

Trong giai đoạn đầu, các thành viên ban đầu đề xuất và biểu quyết để hình thành bộ câu hỏi đầu tiên. Bộ câu hỏi này là nền tảng cho các phiên rà soát và cập nhật về sau.

\subsection{Thay đổi cơ chế phân phối token}

Hiện tại, tất cả người kiểm duyệt trong một phiên kiểm duyệt chất lượng thử thách đều có cơ hội nhận phần thưởng, bất kể kết quả kiểm duyệt có hợp lý hay không.
Tuy nhiên, sau khi đề xuất về việc thành lập hội đồng kiểm soát hành vi được đưa ra, cơ chế phân phối thưởng cần được điều chỉnh để đảm bảo công bằng và nâng cao chất lượng kiểm duyệt.
Cụ thể:
\begin{itemize}
  \item Chỉ những người kiểm duyệt có bài đánh giá được hội đồng kiểm soát xác nhận là \textbf{``hợp lý''} mới được nhận thưởng.
  \item \textbf{Người đại diện hội đồng kiểm soát}, người chịu trách nhiệm đánh giá hành vi trong phiên kiểm duyệt, sẽ được nhận một phần thưởng nhỏ từ quỹ thưởng chung.
\end{itemize}

Tỷ lệ phân phối được đề xuất như sau:
\begin{itemize}
  \item \textbf{10\%} tổng quỹ thưởng dành cho phiên kiểm duyệt sẽ được trao cho người đại diện hội đồng kiểm soát.
  \item \textbf{90\%} còn lại sẽ được phân phối đều cho những người kiểm duyệt có bài đánh giá hợp lý.
\end{itemize}

\subsection{Khấu trừ lợi nhuận từ phí tham gia thử thách}

Hiện tại, người đóng góp nhận toàn bộ phí tham gia thử thách từ người học. Tuy nhiên, để hỗ trợ các hoạt động duy trì nền tảng và khuyến khích sự tham gia của các thành viên có vai trò quan trọng trong quy trình kiểm duyệt, chúng tôi đề xuất cơ chế khấu trừ một phần lợi nhuận này như sau:
\begin{itemize}
  \item \textbf{10\%} dành cho \textbf{người đại diện hội đồng kiểm soát hành vi}.
  \item \textbf{10\%} dành cho \textbf{những người kiểm duyệt có bài đánh giá hợp lý}, được phân phối theo cơ chế quỹ thưởng đã nêu ở mục trước.
  \item \textbf{1\%} chuyển vào \textbf{quỹ chi phí vận hành hệ thống}.
  \item \textbf{79\%} còn lại sẽ được chuyển cho \textbf{người đóng góp}.
\end{itemize}
Cơ chế phân phối này nhằm đảm bảo hệ sinh thái vận hành công bằng, minh bạch và có động lực lâu dài cho các thành phần cốt lõi của nền tảng.

\subsection{Chọn ra những người kiểm duyệt hoặc người đánh giá}

Hiện tại, hệ thống cho phép người kiểm duyệt và người đánh giá có thể chủ động tham gia vào thử thách hoặc giải pháp mong muốn nếu đạt đủ điểm uy tín. 
Điều này dẫn đến khả năng một người dùng có thể sử dụng nhiều ví điện tử để tham gia, làm giảm tính khách quan và độ tin cậy của kết quả cũng như trục lợi từ quỹ treo thưởng.

Do đó, chúng tôi đề xuất cơ chế \textbf{chọn ngẫu nhiên 3 người kiểm duyệt hoặc người đánh giá} có uy tín chuyên môn phù hợp với lĩnh vực chuyên môn của thử thách hoặc giải pháp. 
Cơ chế này không chỉ đảm bảo tính công bằng mà còn tránh tình trạng người dùng chủ động chọn thử thách để thao túng kết quả.

Bên cạnh đó, hệ thống cũng sẽ triển khai một số kỹ thuật kiểm soát hành vi nhằm \textbf{phát hiện người dùng sử dụng nhiều ví điện tử}, chẳng hạn như:
\begin{itemize}
  \item Yêu cầu người dùng liên kết các ví phụ với một ví chính thông qua chữ ký xác thực.
  \item Phân tích hành vi và lịch sử tương tác của các ví để phát hiện các mẫu hành vi trùng lặp.
  \item Áp dụng giới hạn số lượng ví có thể liên kết với một danh tính.
\end{itemize}

Những biện pháp này giúp đảm bảo việc chọn ngẫu nhiên không vô tình chọn nhiều địa chỉ thuộc cùng một cá nhân, từ đó duy trì tính khách quan và độ tin cậy của hệ thống.

\subsection{Một số hướng phát triển khác}

\begin{itemize}
  \item \textbf{Kiểm soát người dùng chưa đăng ký}: Bổ sung cơ chế giới hạn quyền thực hiện các hành vi quan trọng đối với người dùng chưa hoàn tất đăng ký thông tin cá nhân.
  \item \textbf{Kiểm soát quyền truy cập không gian nhà tuyển dụng}: Xây dựng kênh xác thực riêng giữa nhà tuyển dụng và hệ thống, từ đó cho phép chỉ những địa chỉ ví đã được xét duyệt mới có quyền truy cập không gian này.
  \item \textbf{Mở rộng danh mục lĩnh vực chuyên môn}: Phân tách các lĩnh vực hiện có thành các chuyên ngành cụ thể hơn (ví dụ: Hệ thống nhúng, An toàn phần mềm, DevOps), từ đó nâng cao độ chính xác trong đánh giá và phân quyền.
  \item \textbf{Tối ưu hóa mô tả thử thách và giải pháp}: Bổ sung khả năng nhập công thức toán học, biểu đồ, và tải tập tin trực tiếp từ thiết bị để giúp người dùng trình bày đầy đủ và trực quan hơn.
  \item \textbf{Phát triển không gian bình luận cho thử thách}: Mỗi thử thách nên có khu vực phản hồi công khai, nơi cộng đồng có thể thảo luận, đánh giá và chia sẻ trải nghiệm thực tế, góp phần phản ánh chất lượng thử thách.
  \item \textbf{Tích hợp môi trường lập trình vào không gian giải pháp}: Hỗ trợ trình biên dịch theo nhiều ngôn ngữ lập trình phổ biến, giúp người học có thể viết, chạy thử, và hiển thị kết quả ngay trong giao diện hệ thống.
  \item \textbf{Cơ chế phản hồi đánh giá và kiểm duyệt}: Mở rộng chức năng phản hồi từ người đánh giá hoặc kiểm duyệt để gửi nhận xét cụ thể đến người đóng góp hoặc người học, tăng cường tính minh bạch và tính hướng dẫn.
  \item \textbf{Bổ sung ràng buộc thời gian}: Đặt giới hạn thời gian cụ thể cho việc hoàn tất kiểm duyệt và đánh giá nhằm tránh tình trạng trì hoãn và giúp quy trình vận hành trơn tru hơn.
  \item \textbf{Mở rộng khả năng tương tác cộng đồng}: Cho phép người dùng truy cập thông tin hồ sơ của người khác (nếu công khai), tìm kiếm người cố vấn và tham gia không gian cộng đồng để trao đổi và học hỏi lẫn nhau.
  \item \textbf{Nâng cao bảo mật dữ liệu}: Áp dụng mã hóa cho dữ liệu nhạy cảm (on-chain và off-chain), đồng thời tăng cường kiểm soát quyền truy cập để đảm bảo an toàn thông tin cho người dùng.
  \item \textbf{Phát triển ứng dụng di động}: Xây dựng phiên bản dành cho thiết bị di động để mở rộng khả năng tiếp cận hệ thống ở nhiều hoàn cảnh sử dụng khác nhau.
\end{itemize}